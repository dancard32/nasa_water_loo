\pagebreak
\chapter[Fast Actuating Rectum Transfer Exit Device (FARTED)]{}\vspace{-2cm}\noindent\rule{\textwidth}{2.5pt}
\thispagestyle{empty}

\vspace{5cm}\textbf{\huge{Fast Actuating Rectum Transfer Exit Device (FARTED)}}

\medskip\noindent\rule{\textwidth}{1pt}

Short description of design

\pagebreak
\fancyhead[L]{Chapter \thechapter. Waste Collection System}
\section[Overview of Design]{Overview of Waste Collection System}

    \subsection{Working in Microgravity}
    Please discuss in detail how your design will work in both microgravity and lunar gravity

    \subsection{Implementation for All Genders}
    Please discuss in detail how your design will accommodate female and male crew

    \subsection{Design Versatility}
    Please discuss in detail how your design be easy to use and maintain, with low noise

    \subsection{Transferring Collected Waste to Storage}
    Please discuss in detail how your design allow for transfer of collected waste to storage or external vehicle disposal

\pagebreak
\section[Accommodations]{Accommodations of Waste Collection System}

    \subsection{Waste Containment}
    Please discuss in detail how your design will capture and contain urine, feces, vomit, diarrhea, and menses,

    \subsection{Stabilizing Urination}
    Please discuss in detail how your design will Stabilize urine
    
    \subsection{Functionality of Waste Containment System}
    Please discuss in detail how your design will accommodate simultaneous urination and defecation,

    \subsection{Lifetime}
    Please discuss in detail how your design will accommodate the needs of 2 crew members for 14 days,

    \subsection{Integration with Hygiene Products}
    Please discuss in detail how your design will accommodates the use of toilet hygiene products,

    \subsection{Time of Turn Around}
    Please discuss in detail how your design will clears previous waste content prior to next use,

    \subsection{Mission Lifetime}
    Please discuss in detail how your design will defines how often the collections system must be replaced or disposed of in the mission

\pagebreak
\section[Safety]{Safety Measures of Design}

    \subsection{Minimizing Contact with Waste}
    Please discuss the safety measures in place to ensure that during nominal use or in the event of a system failure crew handling of waste materials during maintenance or system use is minimized

    \subsection{Isolating System from Vacuum}
    Please discuss the safety measures in place to ensure that during nominal use or in the event of a system failure crew members are not exposed to vacuum

\pagebreak
\section{Technical Maturity}
    Please discuss the technical maturity of your proposed toilet design. What TRL would you assign it? Please provide a supporting rationale and/or evidence for this rating. Why do you believe this could be developed and integrated into a lunar rover in the next 2-3 years?

\pagebreak
\section{Technical Judging}

\begin{itemize}
    \item  Proposal Quality
    Quality of proposal: clear, concise writing;  thoughtful and complete explanations of how the proposed toilet design concept meets the specifications listed; accompanying CAD file (or other file format) is clear and complete. [10 pts]

    \item Capabilities - Usage
    
    Overall technical feasibility of the proposal toilet design.

    Compatible for use by both female and male crew members.

    How well does the design address issues like ease of use, odor control, noise, and turnaround time.

    Likelihood that it will function in both microgravity and lunar gravity when prototyped.

    How easy will it be to adapt the design for integration into a lunar rover. [20 pts]

    \item Capabilities - Capacity
    Likelihood that it can successfully meet the performance specifications when prototyped, capturing:
    \begin{itemize}
        \item Urine (and stabilizing it)
        \item Feces (accommodating simultaneous urination and defecation)
        \item Diarrhea
        \item Vomit
        \item Menses
    \end{itemize}

    Likelihood that it can accommodate the needs of 2 crew members for 14 days.
        
    Defines how often the collections system must be replaced or disposed of in the mission [20 pts]

    \item Technical Maturity
    The likelihood that proposed toilet design can be developed and integrated in the next 2-3 years.

    Quality of the explanation and supporting evidence for why a solution is designated at a particular maturity level. [20 pts]

    \item Safety
    Confidence that proposed design will minimize the crew handling of crew waste during maintenance or system use in all mission environments, and will not expose crew to vacuum in the event of a system failure. [20 pts]

    \item Innovation
    Novelty or creativity of proposed approach.

    Elegance of design.
    
    Describe how the innovation overcomes limitations and constraints of existing technologies or commercial products. [10 pts]
\end{itemize}

