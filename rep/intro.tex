\pagebreak
\chapter[Introduction]{}\vspace{-2cm}\noindent\rule{\textwidth}{2.5pt}
\thispagestyle{empty}

\vspace{5cm}\textbf{\huge{Introduction}}

\medskip\noindent\rule{\textwidth}{1pt}

As NASA prepares to return to the Moon, innumerable activities to equip, shelter, and otherwise support future astronauts are underway.  These astronauts will be eating and drinking, and subsequently urinating and defecating in microgravity and lunar gravity.  While astronauts are in the cabin and out of their spacesuits, they will need a toilet that has all the same capabilities as ones here on Earth. NASA is calling on the global community for their novel design concepts for compact toilets that can operate in both microgravity and lunar gravity.  These designs may be adapted for use in the Artemis lunar landers that take us back to the Moon.  Although space toilets already exist and are in use (at the International Space Station, for example), they are designed for microgravity only.  NASA’s Human Landing System Program is looking for a next-generation device that is smaller, more efficient, and capable of working in both microgravity and lunar gravity. This challenge includes a Technical category and Junior category.

\pagebreak
It is in this calling of this unique challenge that we wish to provide a solution that is both functional and innovative. This document will provide our design rationale, the functionality of the product, and how this design has innovated prior designs.  

\section[Space Waste Collection Systems]{Overview of Space Waste Collection Systems}
Space Waste Collection Systems(WCS) are the implementation of Earth toilets in a microgravity environment. However, easier said than done while in microgravity there poses a challenge of transporting and storing human waste. While in microgravity, fluids are pulled together through its surface tension causing the mundane task of transporting anything through a fluid system a non-trivial task. As a result of this non-trivial task, this seemingly simple task on Earth becomes a very intensive engineering project that cannot be overlooked.


    \subsection{Apollo Waste Collection System}

    \subsection{Space Shuttle's Waste Collection System}

    \subsection{Soyuz Waste Collection System}

    \subsection{International Space Station Waste Collection System}

    Test \autocite{iss_background}

 
    \pagebreak
    \vspace{-3em}
{
    \printbibliography[title = {References}]
    \addcontentsline{toc}{section}{References}{\vspace{-1cm}\vskip-\ht\strutbox\makebox[\textwidth][c]{\rule{\dimexpr\textwidth}{2pt}}\par}

    \thispagestyle{empty}
}

