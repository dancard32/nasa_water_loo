\pagebreak
\chapter[Lunar Loo Challenge]{}\vspace{-2cm}\noindent\rule{\textwidth}{2.5pt}
\thispagestyle{empty}

\vspace{5cm}\textbf{\huge{Lunar Loo Challenge}}

\medskip\noindent\rule{\textwidth}{1pt}

Artemis is NASA’s program to land the first woman and the next man on the Moon by 2024.  Humanity is going back to the Moon to establish a presence that will enable eventual crewed journeys to Mars.  As we prepare for our return to the Moon, innumerable activities to equip, shelter, and otherwise support future astronauts are underway.  These astronauts will be eating and drinking, and subsequently urinating and defecating in microgravity and lunar gravity.  While astronauts are in the cabin and out of their spacesuits, they will need a toilet that has all the same capabilities as ones here on Earth.  

This challenge seeks to radically change current lunar space toilets to reflect the technological advances that NASA wishes to achieve throughout Artemis's mission timeline.

\pagebreak

\section{Challenge Overview}

\begin{figure}[h]
    \centering
    \includegraphics[width = \linewidth]{figs/lunarloologo.jpg}
    \caption[NASA's Lunar Loo Challenge]{NASA's Lunar Loo challenge, hosted by a collaboration between  hero\textsuperscript{x} and NASA Tournament Lab (NTL)}
\end{figure}

NASA is calling on the global community for their novel design concepts for compact toilets that can operate in both microgravity and lunar gravity.  These designs may be adapted for use in the Artemis lunar landers that take us back to the Moon.  Although space toilets already exist and are in use (at the International Space Station, for example), they are designed for microgravity only.  NASA is looking for a next-generation device that is smaller, more efficient, and capable of working in both microgravity and lunar gravity.  Getting back to the Moon by 2024 is an ambitious goal, and NASA is already working on approaches to miniaturize and streamline the existing toilets.  But they are also inviting ideas from the global community, knowing that they will approach the problem with a mindset different from traditional aerospace engineering.  This challenge hopes to attract radically new and different approaches to the problem of human waste capture and containment.

\pagebreak
\section{Lunar Loo Requirements}
Driving new technological advances often times result from system requirements and the Lunar Loo challenge is no exception. Thought to be a trivial task, designing a toilet that offers services accommodating to many while in microgravity does indeed require attention to detail and knowledge of the microgravity environment.

    \subsection{Toilet Design Specifications}

    The specifications listed below represent the maximum allowed values.  Proposed designs should at least meet them and will preferentially be lower than them.  The toilet design should:

    \begin{itemize}
        \item Function in both microgravity and lunar gravity
        \item Have a mass of less than 15 Kg in Earth’s gravity
        \item Occupy a volume no greater than 0.12 $\text{m}^3$
        \item Consume less than 70 Watts of power
        \item Operate with a noise level less than 60 dB (an average bathroom fan)
        \item Accommodate both female and male users
        \item Accommodate users ranging from 58 to 77 inches tall and 107 to 290 lbs in weight
    \end{itemize}
    
    \pagebreak
    \subsection{Toilet Performance Specifications}

    We are looking for a design that captures all the functionality of a toilet on Earth.  At a minimum, crew using lunar toilets should not be exposed to vacuum during use, and toilet designs should be able to:

    \begin{itemize}
        \item Accommodate simultaneous urination and defecation
        \item Collect up to 1 liter of urine per use, with an average of 6 uses per crew per day
        \item Accommodate 500g of fecal matter per defecation, with an average of 2 uses per crew per day
        \item Accommodate 500g of diarrhea per event
        \item Accommodate an average of 114g of female menses, per crew per day
        \item Stabilize urine to avoid the generation of gas and particulates
        \item Accommodate crew use of toilet hygiene products, like toilet paper, wipes, and gloves
        \item Be clear of previous user’s urine and feces in preparation for the next use
        \item Allow for transfer of collected waste to storage  and/or provide for external vehicle disposal. Minimal Lander volume requires regularly minimizing waste storage or removing it from the vehicle
        \item Allow for easy cleaning and maintenance, with 5 minute turnaround time or less between uses
    \end{itemize}

    Additionally, in the event of system failure, the design will ensure that:

    \begin{itemize}
        \item All waste materials collected remain safely stored
        \item The crew is not exposed to urine, feces, or other collected materials
        \item The crew is not exposed to vacuum
    \end{itemize}



