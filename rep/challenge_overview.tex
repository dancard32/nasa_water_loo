\pagebreak
\chapter[Lunar Loo Challenge]{}\vspace{-2cm}\noindent\rule{\textwidth}{2.5pt}
\thispagestyle{empty}

\vspace{5cm}\textbf{\huge{Lunar Loo Challenge}}

\medskip\noindent\rule{\textwidth}{1pt}

\lipsum[1]

\pagebreak

\section{Challenge Overview}


\section{Lunar Loo Requirements}

    \subsection{Toilet Design Specifications}

    The specifications listed below represent the maximum allowed values.  Proposed designs should at least meet them and will preferentially be lower than them.  The toilet design should:

    \begin{itemize}
        \item Function in both microgravity and lunar gravity
        \item Have a mass of less than 15 Kg in Earth’s gravity
        \item Occupy a volume no greater than 0.12 m3
        \item Consume less than 70 Watts of power
        \item Operate with a noise level less than 60 decibels (no louder than an average bathroom fan)
        \item Accommodate both female and male users
        \item Accommodate users ranging from 58 to 77 inches tall and 107 to 290 lbs in weight
    \end{itemize}
    
    \subsection{Toilet Performance Specifications}

    We are looking for a design that captures all the functionality of a toilet on Earth.  At a minimum, crew using lunar toilets should not be exposed to vacuum during use, and toilet designs should be able to:

    \begin{itemize}
        \item Accommodate simultaneous urination and defecation
        \item Collect up to 1 liter of urine per use, with an average of 6 uses per crew per day
        \item Accommodate 500g of fecal matter per defecation, with an average of 2 uses per crew per day
        \item Accommodate 500g of diarrhea per event
        \item Accommodate an average of 114g of female menses, per crew per day
        \item Stabilize urine to avoid the generation of gas and particulates
        \item Accommodate crew use of toilet hygiene products, like toilet paper, wipes, and gloves
        \item Be clear of previous user’s urine and feces in preparation for the next use
        \item Allow for transfer of collected waste to storage  and/or provide for external vehicle disposal. Minimal Lander volume requires regularly minimizing waste storage or removing it from the vehicle
        \item Allow for easy cleaning and maintenance, with 5 minute turnaround time or less between uses
    \end{itemize}

    Additionally, in the event of a system failure, the toilet designs will ensure that:

    \begin{itemize}
        \item All waste materials collected remain safely stored
        \item The crew is not exposed to urine, feces, or other collected materials
        \item The crew is not exposed to vacuum
    \end{itemize}



